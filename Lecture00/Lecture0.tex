\documentclass[11pt]{amsart}
\usepackage{geometry}                % See geometry.pdf to learn the layout options. There are lots.
\usepackage{graphicx}
\usepackage{amssymb}
\usepackage{epstopdf}
\usepackage{algorithmic}
\usepackage{algorithm}
\usepackage{url}
\usepackage{framed}
 

\title{Lecture 0  - Introduction }
%\date{}                                           % Activate to display a given date or no date

\begin{document}
\maketitle

\section{Course Introduction}
\begin{itemize}
	\item What is Machine Learning? 
	\item Review Syllabus: Course Topics, Grading, Office Hours, etc.
	\item The class website is found on GitHub and Canvas
	\begin{itemize}
		\item Canvas: Grades will be posted here. 
		\item GitHub: Course notes, assignments, projects, extra credit, etc.. will be posted here. 
	\end{itemize}
	\item This tends to be a challenging course.  Suggestions to help you do well: 
	\begin{itemize}
		\item Do the reading in advance and stay on top of your assignments, many topics build on each other 
		\item Ask questions when you do not understand. Ask questions in class, during office hours, or using the Canvas discussion page.  Whenever we are asked a question that may be relevant to others, we will post the question and our response to the Canvas discussion page. 
		\item Participate in classroom/wiki discussion and activities
		\item You are expected to have a computer available that has git installed, the ability to code and compile in python 3+, and run jupyter notebooks.  If attending class in person, bring this computer to class with you so you can follow along. 
	\end{itemize}
	\item Extra Credit: If you find (meaningful) errors or typos in the lecture notes, code, examples, etc posted in the class materials, you can report these using GitHub's ``pull request'' to get some extra credit.  See: \url{https://help.github.com/articles/about-pull-requests/}.  For errors/typos in any PDF file, be sure to correct the associated .tex for my review. 
\end{itemize}
\section{Reading Assignment - Due Aug 28 @ 10:40am}
\begin{itemize}
\item Read Introduction to Chapter 1 and Sections 1.1-1.4
\item Review vectors, matrices and the Gaussian distribution. Some (optional) sources for reviewing this material are:
\begin{itemize} 
\item Chapters 1 \& 2 of Strang's Introduction to Linear Algebra text 
\begin{itemize}
	\item book website: \url{http://math.mit.edu/~gs/linearalgebra/}
	\item associated video lectures: \\\url{https://ocw.mit.edu/courses/mathematics/18-06-linear-algebra-spring-2010/} \end{itemize}
\item Appendix B from Bishop's Pattern Recognition and Machine Learning text.  
\end{itemize}
\end{itemize}

\section{Homework 0 - Due Aug 26 @ 11:59pm}
\begin{enumerate}
\item If you do not already have them, install (1) an editor, (2) Python 3+, (3) the NumPy, SciPy, and Matplotlib packages and (4) Jupyter notebooks on the computer(s) you will be bringing to class and using to complete your assignments.
\begin{itemize}
\item You are welcome to use whatever editor you would like.  There are many options. See: \url{https://wiki.python.org/moin/PythonEditors}  Lately, I have been using SublimeText (which is what I will be using in class).
\item An easy way to install Python 3+ and all required packages is to install Anaconda: \url{https://www.continuum.io/downloads} 

\end{itemize}
\item If you are not familiar with Python 3, go through one (or a few) introductory tutorials:
\begin{itemize}
\item Dive into Python 3: \url{http://www.diveintopython3.net}
\item A list of many beginner tutorials: \url{https://wiki.python.org/moin/BeginnersGuide/Programmers}
\item For those familiar with Matlab, see: \url{http://scipy.github.io/old-wiki/pages/NumPy_for_Matlab_Users.html}
\end{itemize}
\item If you do not have one, create a github account. 
\begin{itemize}
\item Please set up a username that clearly indicates who you are (e.g., my username is alinazare)  
\item Check out: \url{https://education.github.com/pack}
\end{itemize}
\item If you are not familiar with git, complete one (or a few) introductory tutorials:
\begin{itemize}
\item Git bootcamp: \url{https://help.github.com/categories/bootcamp/}
\item Tutorials: \url{https://www.atlassian.com/git/tutorials/}
\item Interactive Introduction: \url{https://try.github.io/}
\end{itemize}
\item Follow the assignment link for the Homework00 repository.  After accepting the invitation, clone the repository to your local machine. 

\item Code up a python program that prints the following:

\begin{itemize}
\item My name is [your name].
\item I am a [undergraduate/masters/PhD] student majoring in [your major] in the Department of [your department].
\item I have [no/some/extensive] programming experience in general.
\item I have [no/some/extensive] python programming experience.
\item I have [no/some/extensive] experience with using version control software. 
\item I have [no/some/extensive] experience with git.
\item I have [no/some/extensive] experience in machine learning.
\item My github username is [username].
\item I am excited to learn the following in this class: [complete with your response]
\end{itemize}
\item Starter code has been provided for you in your homework repository.  You only need to edit, verify that it runs properly, and push back your submission. 
\item Submit this through GitHub by pushing your completed code back to the github assignment repository. 
\end{enumerate}


\end{document}  